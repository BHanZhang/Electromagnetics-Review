\documentclass[UTF8,AutoFakeBold,b5paper]{ctexbook}
\usepackage{ctex}
\usepackage{framed}
\usepackage{amsthm}
\usepackage{geometry}
\usepackage{amsthm,amsmath,amssymb}
\usepackage{mathrsfs}
\geometry{left=2.0cm,right=2.0cm,top=2.0cm,bottom=2.0cm}
\usepackage{amsmath}
\usepackage{graphicx}
\usepackage{subfiles}
\usepackage{color}
\title{\kaishu\textbf {电磁学复习}}
\author{\kaishu 张博涵(xb782053@gmail.com)\\
\kaishu bilibili:天才我张主教\\
\kaishu 知乎:小张也要开心\\
\kaishu GitHub:www.github.com/BHanZhang}
\date{\kaishu \today}
\setCJKsansfont{KaiTi}
\usepackage{chemfig}
\usepackage{mathrsfs}
\usepackage{listings}
\usepackage{makeidx}
\makeindex
\usepackage{framed}
\usepackage{amsthm,amsmath,amssymb}
\usepackage{wrapfig}
\usepackage{graphicx}
\usepackage{mathrsfs}
\bibliographystyle{plain}
\usepackage{subfiles}
\usepackage{booktabs}
\usepackage{graphicx,times}
\usepackage{esint}
\usepackage{times}
\usepackage{subfigure}         
\usepackage{natbib}
\usepackage{amssymb,amsmath}
\usepackage{url}
\usepackage{geometry}
\usepackage{xcolor}
\usepackage{setspace}
\usepackage{subfigure}
\usepackage{tikz}
\everymath{\displaystyle}
\usepackage{booktabs}
\usepackage{array}
\usepackage{mhchem}
%\usepackage[usenames,dvipsnames]{color}
\usepackage{colortbl}
\usepackage{bm}

\definecolor{mygray}{gray}{.9}
\definecolor{mypink}{rgb}{.99,.91,.95}
\definecolor{mycyan}{cmyk}{.3,0,0,0}


\usepackage[breaklinks,colorlinks,linkcolor=black,citecolor=black,urlcolor=black]{hyperref}
\begin{document}
	\maketitle
	\tikz[remember picture, overlay] 
  \node at (current page.center) {\kaishu{祝大家取得好成绩!}};
	\kaishu
	\tableofcontents
	

\chapter{静电场}
\section{Coulomb定律}
电荷的量值是量子化的,其中最基本单位称之为元电荷,元电荷$e = 1.602 \times 10^{-19}$C


\subsubsection{Coulomb定律}\index{Coulomb定律}
在真空中,两个静止的点电荷$q_1$与$q_2$之间的相互作用力的大小和$q_1$与$q_2$的乘积成正比,和他们的距离$r$的平方成反比;作用力的方向沿着他们的联线,同号排斥,异号吸引。

用$\bm{F}_{12}$代表$q_1$对$q_2$的力,$r$代表两电荷之间的距离,$\bm{\hat{r}}_{12}$代表$q_1$到$q_2$方向的单位矢量,即:
\textcolor[rgb]{0.54,0.13,0.33}{
\begin{equation}
	\bm{F}_{12} = k\dfrac{q_{1}q_{2}}{r^{2}}\bm{\hat{r}}_{12}
\end{equation} }

\section{电场强度矢量}\index{电场强度}
\subsubsection{电场强度矢量定义}
若记试探电荷$q_0$在电场某处受力$\bm{F}$,那么电场该处的电场强度$\bm{E}$为:
\begin{equation}
	\bm{E} = \dfrac{\bm{F}}{q_{0}}
\end{equation} 
其大小等于单位电荷在该处收电场力的大小;其方向与正电荷在该处所受电场力方向一致。

如果电场空间各点处场强方向和大小都相同,这种电场即为匀强电场。\index{匀强电场}

\fangsong \textcolor[rgb]{0.07,0.36,0.57}{例题:求点电荷$q$在空间某点产生的场强}

\textcolor[rgb]{0.54,0.13,0.33}{解:非常简单,由
\begin{equation}
	\bm{F} = \dfrac{1}{4\pi\varepsilon_0}\dfrac{qq_0}{r^{2}}\bm{\hat{r}}
\end{equation}
然后根据定义式可知
\begin{equation}
	\bm{E} = \dfrac{\bm{F}}{q_0}=\dfrac{1}{4\pi\varepsilon_0}\dfrac{q}{r^{2}}\bm{\hat{r}}
	\label{2}
\end{equation}
}\index{点电荷产生的场强}

\subsubsection{电偶极子}\kaishu\index{电偶极子}
定义一对靠的很近的等量异号电荷构成的带电体系为电偶极子。
等量异号电荷距离为$l$,其中点到电场某处的距离为$r$,相关的电场结论为:
\begin{equation}
	\text{电偶极子沿长线上:}E \approx \dfrac{1}{4\pi\varepsilon_{0}}\dfrac{2p}{r^{3}}
\end{equation}

\begin{equation}
	\text{电偶极子中垂面上:}E \approx \dfrac{1}{4\pi\varepsilon_{0}}\dfrac{p}{r^{3}}
\end{equation}
其中$p$为电偶极矩,关于其定义为:
\begin{equation}
	\bm{p} = q\bm{l}
\end{equation}
\textcolor[rgb]{0.56,0.28,0.16}{电偶极子这个带电系统和点电荷这个带电系统相比,其场强衰减速率明显加快;且场强与$q$与$l$都有关,如此符合常识。}

\subsubsection{电荷的密度}
设某个体积元$ \Delta V $包含$P$点,设在$ \Delta V $中全部电荷的代数和为$\sum q$,定义$P$点电荷的体密度为:\index{点电荷的体密度}\index{面电荷的体密度}\index{线电荷的体密度}
\begin{equation}
	\rho_{e} = \lim_{\Delta V \to 0}\dfrac{\sum q}{\Delta V}
\end{equation}

类似定义了某点处电荷的面密度(面元$\Delta S$)、线密度(线元$\Delta l$):

\begin{equation}
	\sigma_{e} = \lim_{\Delta S \to 0}\dfrac{\sum q}{\Delta S}
\end{equation}

\begin{equation}
	\eta_{e} = \lim_{\Delta l \to 0}\dfrac{\sum q}{\Delta l}
\end{equation}
\subsubsection{电场的叠加}\index{电场的叠加}
很明显,电场强度矢量$\bm{E}$是个矢量,符合向量空间的八条运算律,可以直接进行矢量叠加。

带均匀电荷且线电荷密度为$\eta_{e}$的细棒中垂线上的场强结论为:若记细棒中垂线上某点到细棒的距离为$r$,细棒长度为$l$则有:
\textcolor[rgb]{0.56,0.28,0.16}{\begin{equation}
	E = \dfrac{\eta_{e}l}{2\pi\varepsilon_{0}r\sqrt{r^{2}+l^{2}}}
\end{equation}}
如果是无限长的细棒,则有:\index{无限长的细棒的电场强度}
\textcolor[rgb]{0.56,0.28,0.16}{
\begin{equation}
	E = \lim_{l \to \infty}\dfrac{\eta_{e}l}{2\pi\varepsilon_{0}r\sqrt{r^{2}+l^{2}}} = \dfrac{\eta_{e}}{2\pi\varepsilon_{0}r}
\end{equation}}
\subsubsection{电偶极子的力矩}\index{电偶极子的力矩}
简单的矢量分析,并利用力矩的叠加原理,可以知道力矩$\bm{M}$:
\textcolor[rgb]{0.56,0.28,0.16}{\begin{equation}
	\bm{M} = \bm{p} \times \bm{E}
\end{equation}}

\section{静电场Gauss定理}\index{静电场Gauss定理}
\subsubsection{立体角}\index{立体角}
类比弧度(rad)的概念,定义立体角(sr):
\begin{equation}
	\ce{d}\Omega = \dfrac{\ce{d}S}{r^{2}}\ce{sr} 
\end{equation}
常时要研究空间某点对于任意一个原点张成的立体角,公式可以进行推广:
\begin{equation}
	\ce{d}\Omega = \dfrac{\hat{\bm{r}}\cdot \ce{d}\bm{S}}{r^{2}}\ce{sr} 
\end{equation}
其中$\hat{r}$为单位径矢,$\ce{d}\bm{S}$为面元。

很明显可以看出:对于一个闭合曲面来讲,某点的立体角为 $4\pi$即:
\begin{equation}
	\varoiint_{S}\ce{d} \Omega = 4\pi
	\label{1}
\end{equation}
\subsubsection{电通量}
定义经过面元$\ce{d}\bm{S}$的电通量:
\begin{equation}
	\ce{d}\Phi_{E} = \bm{E} \cdot \ce{d} \bm{S}
\end{equation}
\subsubsection{Gauss定理}
通过一个任意闭合的曲面$S$的电通量$\Phi_{E} $等于该面包围的所有电量的代数和$\sum q$除以$\varepsilon_{0}。$即:\textcolor[rgb]{0.54,0.13,0.33}{
\begin{equation}
	\Phi_{E} = \varoiint_{S} \bm{E}\cdot \ce{d}\bm{S} = \dfrac{\displaystyle\sum_{S\text{内}}q}{\varepsilon_{0}}
\end{equation}}

证明即分「电荷在闭合曲面内」和「闭合曲面外」两种情况,当在「闭合曲面外」,显然电通量为 $0$,;当在「闭合曲面内」,带入点电荷的公式\ref{2},并找出立体角的方面,积分进去,利用性质\ref{1},直接得到 Gauss 定理。
再补充说明一下关于向量内积的性质,就可以处理多个电荷在曲面内的情况。从略。
\subsubsection{均匀带电球壳产生的电场}\index{均匀带电球壳产生的电场}
设某均匀带电球壳上带电总量为$Q$则有:
\begin{equation}
	\text{球内:} E = 0
\end{equation}
\begin{equation}
	\text{球外:} E = \dfrac{1}{4\pi \varepsilon_{0}}\dfrac{Q}{r^{2}}
\end{equation}
\textcolor[rgb]{0.56,0.28,0.16}{可以看出,均匀带电球壳在空间中产生的场强与其半径毫无关系(只是影响了电场强度为 $0$的多少)。
另一方面,我们可以将球壳上的电荷看做集中在球心上,对解题有帮助。}
\subsubsection{均匀带电球体产生的电场}\index{均匀带电球体产生的电场}
设某均匀带电球体的半径为$R$,带电总量为$Q$则有:
\begin{equation}
	\text{球内:} E = \dfrac{1}{4\pi \varepsilon_{0}}\dfrac{Qr}{R^{3}}
\end{equation}
\begin{equation}
	\text{球外:} E = \dfrac{1}{4\pi \varepsilon_{0}}\dfrac{Q}{r^{2}}
\end{equation}
\subsubsection{无限大带电平面的电场}\index{无限大带电平面的电场}
耳熟能详的公式了,常考选择填空,矢量叠加,方向根据常识判断:
\begin{equation}
	E = \dfrac{\sigma_{e}}{2\varepsilon_{0}}
\end{equation}
\section{静电场环路定理和电势}
\subsubsection{静电场的环路定理}\index{静电场环路定理}
试探电荷在任何静电场中移动时,电场力所做的功只与试探电荷大小及起始点、终止点的位置有关与路径无关,即:
\begin{equation}
	\oint \bm{E} \cdot \ce{d}\bm{l} = 0
\end{equation}
该表述等价为\textcolor[rgb]{0.56,0.28,0.16}{“静电场力做功与路径无关,静电场是保守力场”}。
\subsubsection{电势差的定义}\index{电势差}
将电势降落称为电势差,$P$和$Q$点之间的电势差是$Q$点的电势减去$P$点的电势;是从 $P$点到$Q$点移动单位正电荷时电场力所做的功。即:
\begin{equation}
	U_{PQ} = \dfrac{W_{PQ}}{q_{0}} = \int_{P}^{Q}\bm{E}\cdot\ce{d}\bm{l} 
\end{equation}
\subsubsection{电势的定义}\index{电势}
选取某一参考$0$点,比如常选取无穷远点为电势$0$点,则电势定义为:把单位正电荷移动到无穷远点电场力所做的功,\textcolor[rgb]{0.56,0.28,0.16}{因为电场力做功与路径无关,所以移动路径可以任意选取}。
\begin{equation}
	U(P) = U_{P\infty} = \int_{P}^{\infty}\bm{E}\cdot\ce{d}\bm{l} 
\end{equation}
\subsubsection{电势叠加原理}
由于在有限个积分项下,积分号和求和号可以任意交换次序,所以立得叠加原理。
\subsubsection{点电荷产生的电势}
根据定义立得:
\begin{equation}
	U_{i}(P) = \int_{P}^{\infty} \bm{E}_{i}\cdot\ce{d}\bm{l}  =\dfrac{1}{4\pi\varepsilon_{0}}\dfrac{q_{i}}{r_{i}}
\end{equation}
\subsubsection{带电球壳产生的电势}\index{带电球壳产生的电势}
\begin{equation}
	\text{在球壳内:}U = \dfrac{1}{4\pi\varepsilon_{0}}\dfrac{Q}{r}
\end{equation}
\begin{equation}
	\text{在球壳外:}U = \dfrac{1}{4\pi\varepsilon_{0}}\dfrac{Q}{R}
\end{equation}
\subsubsection{电偶极子产生的电势}\index{电偶极子产生的电势}
取一对电偶极子,其在空间任一点产生的电势为:
\begin{equation}
	U =\dfrac{1}{4\pi\varepsilon_{0}}\dfrac{\bm{p}\cdot\hat{\bm{r}}}{r^{2}}
\end{equation}
\subsubsection{电势的梯度}\index{电势的梯度}
因为我们懂得一些基本知识即“数量场的梯度是向量场”,所谓梯度,\textcolor[rgb]{0.56,0.28,0.16}{就是某一函数对空间坐标的偏导数}。不难记住:
\begin{equation}
	\bm{E} = -\nabla U
\end{equation}
\section{静电能}\index{静电能}
貌似只考电容器能量公式就很简单:
\begin{equation}
	W = \dfrac{Q^{2}}{2C}
\end{equation}
能量密度:\index{能量密度}
\begin{equation}
	\omega_{e} = \dfrac{1}{2}DE = \dfrac{1}{2}\varepsilon E^{2}
\end{equation}
其中$\varepsilon$是介电常数,$\varepsilon_{0}$是真空电导率(真空介电常数),$\varepsilon_{r}$是相对介电常数,有这样的关系:\index{介电常数}\index{真空电导率}\index{真空介电常数}\index{相对介电常数}
\begin{equation}
	\varepsilon =\varepsilon_{r}\varepsilon_{0}
\end{equation}
$D$是电位移矢量其定义为:\index{电位移矢量}
\begin{equation}
	D = \varepsilon_{0}\bm{E}+\bm{P} = \varepsilon_{0}\bm{E}+\chi_{e}\varepsilon_{0}\bm{E} = (1+\chi_{e})\varepsilon_{0}\bm{E}
\end{equation}
其中$\bm{P}$是电极化强度\index{电极化强度}(相当于束缚电荷$q'$的电场强度)。$\chi_{e}$是电极化率\index{电极化率} ,并有$\varepsilon_{r} = 1+\chi_{e}$,还有另外一个关系式:
\begin{equation}
	\varoiint_{S} \bm{E}\cdot \ce{d}\bm{S} =\dfrac{1}{\varepsilon_{0}}\displaystyle\sum_{S\text{内}}(q_{0}+q') 
\end{equation}
\begin{equation}
	\oint_{S} \bm{D}\cdot\ce{d}S = \displaystyle\sum_{S\text{内}}q_{0}
\end{equation}
这是完整版的静电场Gauss 定理和环路定理。
\section{电容}
\subsubsection{电容的定义}\index{电容}
考虑一个孤立带电$q$的导体,具有一定的电势$U$,定义电容$C$为:
\begin{equation}
	C = \dfrac{q}{U}
\end{equation}
\subsubsection{孤立导体球的电容}
设此孤立导体球的半径为$R$,则有:
\begin{equation}
		C = \dfrac{q}{U} = 4\pi\varepsilon_{0}R
\end{equation}
\subsubsection{平行板电容器的电容}\index{平行板电容器的电容}
\begin{equation}
	C = \dfrac{\varepsilon_{0}S}{d}
\end{equation}
\subsubsection{同心球电容器的电容}\index{同心球电容器的电容}
两个同心球,半径分别为$R_{A}$、$R_{B}$($R_{A}<R_{B}$)其两球之间的电容为:
\begin{equation}
	C = \dfrac{4\pi\varepsilon_{0}R_{A}R_{B}}{R_{B}-R_{A}}
\end{equation}
\subsubsection{同心柱电容器的电容}\index{同心柱电容器的电容}
两个同心柱,半径分别为$R_{A}$、$R_{B}$($R_{A}<R_{B}$),柱长为$L$,其两球之间的电容为:
\begin{equation}
	C =\dfrac{2\pi\varepsilon_{0}L}{\ce{ln}\dfrac{R_{B}}{R_{A}}}
\end{equation}
\chapter{恒磁场}
\section{B.S定律与Ampere定律}
设$I$是引起磁场的闭合电流。用$I\ce{d}\bm{l}$是任意一个闭合载流回路$L$中的任意一个电流元,则有:\index{B.S定律}
\begin{equation}
	\ce{d}\bm{B} = \dfrac{\mu_{0}}{4\pi}\dfrac{I\ce{d}\bm{l}\times\hat{\bm{r}}}{r^{2}}
\end{equation}
\begin{equation}
	\bm{B} = \oint_{(L)}\ce{d}\bm{B} =\dfrac{\mu_{0}}{4\pi} \oint_{(L)}\dfrac{I\ce{d}\bm{l}\times\hat{\bm{r}}}{r^{2}}
\end{equation}
\textcolor[rgb]{0.56,0.28,0.16}{B.S 定律认为任意闭合回路产生的磁感应强度$\bm{B}$看成是各个电流元$I\ce{d}\bm{l}$产生的元磁场强度的叠加。}
\subsubsection{Ampere定律}
讲道理,Ampere定律 = 安培力公式 + B.S定律:
\begin{equation}\index{Ampere定律}\index{安培力公式}
	\text{安培力公式:}\ce{d}\bm{F}_{12} = I_{2}\ce{d}\bm{l}_{2} \times \ce{d}\bm{B}
\end{equation}
\begin{equation}
	\text{Ampere定律:}\ce{d}\bm{F}_{12} = \dfrac{\mu_{0}}{4\pi}\dfrac{I_{2}\ce{d}\bm{l}_{2} \times(I_{1}\ce{d}\bm{l}_{1} \times \hat{\bm{r}}_{12})}{r^{2}_{12}}
\end{equation}
\subsubsection{载流直导线周围的磁场}\index{载流直导线周围的磁场}
考虑距离电流为$I$的载流直导线$r$处的一点$P$,其磁场为:
\begin{equation}
	B = \dfrac{\mu_{0}I}{4\pi r_{0}}(\cos \theta_{1} -\cos\theta_{2})
\end{equation}
其中$\theta$是积分上下限点和$P$点连线与电流方向的夹角。

当导线为无限长($r_{0} \ll l$)的时候:
\textcolor[rgb]{0.54,0.13,0.33}{
\begin{equation}
	B = \dfrac{\mu_{0}I}{4\pi r_{0}}
\end{equation}}
\textcolor[rgb]{0.56,0.28,0.16}{在无限长载流直导线周围的磁感应强度$B$的大小与距离$r_{0}$的一次方成反比。}
\subsubsection{载流圆线圈轴线上的磁场}\index{载流圆线圈轴线上的磁场}
设半径为$R$的载流圆线圈,距离圆心$r_{0}$的轴线上点$P$的磁场大小为:\textcolor[rgb]{0.54,0.13,0.33}{
\begin{equation}
	B = \dfrac{\mu_{0}R^{2}I}{2(R^{2}+r^{2}_{0})^{3/2}}
\end{equation}}
做一些近似:\textcolor[rgb]{0.54,0.13,0.33}{
\begin{equation}
	\text{在圆心处:}B = \dfrac{\mu_{0}I}{2R}
\end{equation}
\begin{equation}
	r_{0}\gg R\text{ 时:} B = \dfrac{\mu_{0}R^{2}I}{2r^{3}_{0}}
\end{equation}}
\subsubsection{载有环向电流的圆筒在轴线上产生的磁场}\index{载有环向电流的圆筒在轴线上产生的磁场}\index{螺线管}
将绕在圆柱面上的螺线形线圈叫做螺线管。如果绕螺线管的导线很细且密绕,可以把它看做一个导体圆筒。现在在一些特殊情况下来计算这个磁场:
\begin{equation}
	\text{无限长圆筒:} B = \mu_{0}n\iota
\end{equation}
\begin{equation}
	\text{在半无限长圆筒的一侧:} B = \dfrac{\mu_{0}n\iota}{2}
\end{equation}

\section{Ampere环路定理}
磁感应强度沿任何闭合环路$L$的线积分,等于穿过这环路所有电流的代数和的$\mu_{0}$倍,即:
\begin{equation}
	\oint_{(L)}\bm{B}\cdot \ce{d}\bm{l} =\mu_{0}\displaystyle\sum_{L\text{内}}I
\end{equation}

\section{磁场的Gauss定理}
\subsubsection{磁场的Gauss定理}\index{磁场的Gauss定理}
通过一个曲面的磁通量仅由此曲面的边界线决定;通过闭合曲面的磁通量和必定为 $0$,即:
\begin{equation}
	\varoiint_{(S)} \bm{B} \cdot \ce{d}\bm{S} = 0
\end{equation}

\subsubsection{磁矢势}\index{磁矢势}
把闭合曲面$S$分成两部分$S_{1}$、$S_{2}$,这两部分拥有共同边界$L$,并彼此相反。这下有:
\begin{equation}
	\varoiint_{(S)} \bm{B} \cdot \ce{d}\bm{S} = \iint_{(S_{1})} \bm{B} \cdot \ce{d}\bm{S} -\iint_{(S_{2})} \bm{B} \cdot \ce{d}\bm{S}  = 0
	\label{3}
\end{equation}
\textcolor[rgb]{0.56,0.28,0.16}{这下说明磁通量仅由$S_{1}$、$S_{2}$的共同边界$L$所决定。}
定义磁矢势、由上式\ref{3}可得:
\begin{equation}
	\iint_{(S_{1})} \bm{B} \cdot \ce{d}\bm{S} -\iint_{(S_{2})} \bm{B} \cdot \ce{d}\bm{S} = \oint_{(L)}\bm{A} \cdot \ce{d}\bm{l}  
\end{equation}

\chapter{电磁感应}

\section{Faraday电磁感应定律}\index{Faraday电磁感应定律}
\subsubsection{Faraday电磁感应定律}
导体回路中感应电动势$\mathscr{E}$的大小与穿过回路磁通量的变化率$\dfrac{\ce{d}\Phi}{\ce{d}t}$成正比,即:
\begin{equation}
	\mathscr{E}\propto \dfrac{\ce{d}\Phi}{\ce{d}t},\mathscr{E}=-\dfrac{\ce{d}\Phi}{\ce{d}t}
\end{equation}
\section{楞次定律}
\subsubsection{楞次定律}\index{楞次定律}
闭合回路中感应电流的方向总是使得其激发的磁场来阻碍引起感应电流的磁通量的变化。\textcolor[rgb]{0.56,0.28,0.16}{感应电流的效果总是反抗引起感应电流的原因。}
\section{动生电动势和感生电动势}
\subsubsection{动生电动势}\index{动生电动势}
\begin{equation}
	\mathscr{E} = \oint_{(L)} (\bm{v} \times \bm{B}) \cdot \ce{d}\bm{l} = \int_{B}^{A} (\bm{v} \times \bm{B}) \cdot \ce{d}\bm{l}
\end{equation}
\subsubsection{感生电动势}\index{感生电动势}
\begin{equation}
	\mathscr{E} = -\dfrac{\ce{d}\Phi}{\ce{d}t} = -\dfrac{\ce{d}}{\ce{d}t}\iint_{(S)}\bm{B}\cdot \ce{d}\bm{S} = -\dfrac{\ce{d}}{\ce{d}t}\oint_{(\partial S)}\bm{A}\cdot \ce{d}\bm{l}=-\oint_{(\partial S)}\dfrac{\partial\bm{A}}{\partial t} \cdot \ce{d}\bm{l}
\end{equation}
\printindex
\end{document}